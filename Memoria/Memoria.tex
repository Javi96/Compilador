\documentclass[11pt,oneside,onecolumn,openany]{book}

\usepackage{lmodern}
\usepackage[T1]{fontenc}
\usepackage[spanish,activeacute]{babel}
\usepackage{mathtools}

\title{Pr�ctica PL}
\author{Elena Kaloyanova Popova y �lvaro Borja Velasco Garc�a}
\date{2018}

\begin{document}
% cuerpo del documento

\maketitle
\tableofcontents % crea el �ndice

%---------------------------------------------------------------------
%                   Cap�tulo 1 - Introducci�n
%---------------------------------------------------------------------
\chapter{Introducci�n}

Esta pr�ctica consistir� en el desarrollo de un procesador de lenguajes sobre el siguiente lenguaje:

%---------------------------------------------------------------------
%                   Cap�tulo 2 - Analizador l�xico
%---------------------------------------------------------------------
\chapter{Fase 1: Analizador l�xico}

%---------------------------------------------------------------------
\section{Clases L�xicas}
%---------------------------------------------------------------------
\label{cap2:sec:clases_lexicas}
Enumeraci�n de las clases l�xicas del lenguaje. Para cada clase debe incluirse adem�s una descripci�n informal, en lenguaje natural.
Las clases l�xicas que hemos considerado son las siguientes:

\begin{itemize}
\textbf{Programa}: Todo programa consta de dos secciones: una para las declaraciones y otra para las instrucciones, separadas por "`\&\&"' \textbf{PROGRAMA}
\textbf{Declaraciones:} Compuestas por el nombre de tipo y el nombre de variable. \textbf{DECL}
\textbf{Nombre de tipo:} Pueden ser o "`num"' o "`bool"'. \textbf{TIPO}
\textbf{Nombre de variable:} Comienzan necesariamente por una letra, seguida por una secuencia de cero o m�s letras, d�gitos o "`\_"'. \textbf{VAR}
\textbf{Instruccion:} Constan de una variable seguida de un IGUAL seguido por una expresi�n (asignaciones). \textbf{INSTR}
\textbf{Expresiones b�sicas:} Pueden ser n�meros reales con y sin signo, "`true"' y "`false"'. \textbf{BASICA}
\textbf{N�mero:} Pueden empezar opcionalmente con un signo + o - seguidos de una secuencia de uno o m�s digitos cualesquiera, pudiendo poner ceros no significativos a la izquierda. Puede opcionalmente estar seguida por una parte decimal y/o una parte exponencial. \textbf{NUM}
\textbf{Parte decimal:} Consta de un punto seguido por una parte entera. \textbf{DEC}
\textbf{Parte exponencial:} Consta de un letra E may�scula o min�scula seguida, opcionalmente, por un signo + o -, y obligatoriamente por una parte entera. \textbf{EXP}
\textbf{Operadores:} Pueden ser aritm�ticos, l�gicos o relacionales. \textbf{OP}
\textbf{Operadores aritm�ticos:} Pueden ser +,-,*,/. \textbf{OPAR}
\textbf{Operadores l�gicos:} Pueden ser "`and"', "`or"' y "`not"'. \textbf{OPLOG}
\textbf{Operadores relacionales:} Pueden ser [<,>,<=,>=,==,!=]. \textbf{OPREL}
\textbf{Par�ntesis de apertura:} Signo "`C"'. \textbf{PAP}
\textbf{Par�ntesis de cierre:} Signo "`C"'. \textbf{PCIERRE}

\end{itmize}
%---------------------------------------------------------------------
\section{Especificaci�n Formal}
%---------------------------------------------------------------------
\label{cap2:sec:especificacion_formal}
Las definiciones regulares que definen el lenguaje son las siguientes:

\begin{itemize}
	\item \textbf{PROGRAMA:} DECL\&\&INSTR
	\item \textbf{DECL:} (TIPOBVAR;)*|(TIPOBVAR)
	\item \textbf{TIPO:} (num|bool)
	\item \textbf{VAR:} ([a-z,A-Z][a-z,A-Z,0-9,\_]) 
	\item \textbf{INSTR:} VAR=EXPRESION
	\item \textbf{EXPRESION:} BASICA.OP.BASICA | BASICA.OP.COMP | COMP.OP.COMP | COMP.OP.BASICA 
	\item \textbf{BASICA:} [VAR,NUM,true,false]
	\item \textbf{NUM:} SIGNO?DIG+(DEC)?(EXP)? 
	\item \textbf{SIGNO:} [+,-] 
	\item \textbf{DIG:} [0-9] 
	\item \textbf{DEC:} (.DIG+) 
	\item \textbf{EXP:} ([e|E]SIGNO?DIG+) 
	\item \textbf{OP:} [OPAR,OPLOG,OPREL] 
	\item \textbf{OPAR:} [SIGNO,*,/] 
	\item \textbf{OPLOG:} [and,or,not]
	\item \textbf{OPREL:} [<,>,<=,>=,==,!=]
	\item \textbf{SEP:} [BLANCO,TAB,CARRO,ENDL,ENDF]
	\item \textbf{COMP:} BASICA.OP.BASICA
	\item \textbf{PAP:} (C)
	\item \textbf{PCIERRE:} (C)
	
\end{itemize}
%---------------------------------------------------------------------
\section{Dise�o}
%---------------------------------------------------------------------
\label{cap2:sec:disenyo}
Dise�o de un analizador l�xico para el lenguaje mediante un diagrama de transiciones.

\end{document}
