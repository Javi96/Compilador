\documentclass[11pt,oneside,onecolumn,openany]{book}

\usepackage{lmodern}
\usepackage[T1]{fontenc}
\usepackage[spanish,activeacute]{babel}
\usepackage{mathtools}

\title{Pr�ctica PL}
\author{Elena Kaloyanova Popova y �lvaro Borja Velasco Garc�a}
\date{2018}

\begin{document}
% cuerpo del documento

\maketitle
\tableofcontents % crea el �ndice

%---------------------------------------------------------------------
%                   Cap�tulo 1 - Introducci�n
%---------------------------------------------------------------------
\chapter{Introducci�n}

Esta pr�ctica consistir� en el desarrollo de un procesador de lenguajes sobre el siguiente lenguaje:

%---------------------------------------------------------------------
%                   Cap�tulo 2 - Analizador l�xico
%---------------------------------------------------------------------
\chapter{Fase 1: Analizador l�xico}

%---------------------------------------------------------------------
\section{Clases L�xicas}
%---------------------------------------------------------------------
\label{cap2:sec:clases_lexicas}
Todo programa consta de dos secciones: una para las declaraciones y otra para las instrucciones, separadas por un token "`\&\&"'. La secci�n de declaraciones est� formada por una serie de declaraciones compuestas por el nombre de tipo y el de variable y separadas por un punto y coma. La secci�n de instrucciones, por su parte, consta de una serie de asignaciones (variable=expresi�n), separadas tambi�n por un punto y coma.
Las clases l�xicas que hemos considerado para representar los tokens del lenguaje son las siguientes:

\begin{itemize}
	\item \textbf{SEC:} Representa el seccionador de las dos partes del programa ("`\&\&"').
	\item \textbf{NUM:} Palabra reservada "`num"'.
	\item \textbf{BOOL:} Palabra reservada "`bool"'.
	\item \textbf{VAR:} Representa el nombre de la variable. Comienza necesariamente por una letra, seguida por una secuencia de cero o m�s letras, d�gitos o el s�mbolo "`\_"'.
	\item \textbf{ASIG:} Representa el signo igual de las asignaciones.
	\item \textbf{TRUE:} Palabra reservada "`true"'.
	\item \textbf{FALSE:} Palabra reservada "`false"'.
	\item \textbf{NUM:} Representa un n�mero real. Puede empezar opcionalmente con un signo seguido de una secuencia de uno o m�s digitos cualesquiera, pudiendo poner ceros no significativos a la izquierda. Puede opcionalmente estar seguido por una parte decimal y/o una parte exponencial.
	\item \textbf{MAS:} Operador suma (\textbackslash +).
	\item \textbf{MENOS:} Operador resta (\textbackslash -).
	\item \textbf{POR:} Operador multiplicaci�n (\textbackslash *).
	\item \textbf{DIV:} Operador divisi�n (\textbackslash /).
	\item \textbf{AND:} Palabra reservada "`and"'.
	\item \textbf{OR:} Palabra reservada "`or"'.
	\item \textbf{NOT:} Palabra reservada "`not"'.
	\item \textbf{MAY:} Operador mayor (>).
	\item \textbf{MEN:} Operador menor (<).
	\item \textbf{MAYI:} Operador mayor o igual (>=).
	\item \textbf{MENI:} Operador menor o igual (<=).
	\item \textbf{IGUAL:} Operador igual a (==).
	\item \textbf{DIST:} Operador distinto a (!=).
	\item \textbf{PAP:} Signo de apertura de par�ntesis.
	\item \textbf{PCI:} Signo de cierre de par�ntesis.

\end{itemize}
%---------------------------------------------------------------------
\section{Especificaci�n Formal}
%---------------------------------------------------------------------
\label{cap2:sec:especificacion_formal}

Las definiciones regulares correspondientes a las clases l�xicas definidas son:

\begin{itemize}
	\item \textbf{SEC:} \&\&
	\item \textbf{VAR:} LETRA([LETRA|DIG|\textbackslash \_]*)
	\item \textbf{LETRA:} ([a-z,A-Z])
	\item \textbf{NUM:} ([n][u][m])
	\item \textbf{BOOL:} ([b][o][o][l])
	\item \textbf{TRUE:} ([t][r][u][e])
	\item \textbf{FALSE:} ([f][a][l][s][e])
	\item \textbf{NUM:} SIGNO?(DIG+(DEC)?(EXP)?) 
	\item \textbf{SIGNO:} [\textbackslash +,\textbackslash -] 
	\item \textbf{DIG:} [0-9] 
	\item \textbf{DEC:} (\textbackslash .)DIG+ 
	\item \textbf{EX:} [e|E](SIGNO?DIG+(DEC)?)	
	\item \textbf{AND:} ([a][n][d])
	\item \textbf{OR:} ([o][r])
	\item \textbf{NOT:} ([n][o][t])
	\item \textbf{MAS:} (\textbackslash +)
	\item \textbf{MENOS:} (\textbackslash -)
	\item \textbf{DIV:} (\textbackslash /)
	\item \textbf{POR:} (\textbackslash *)
	\item \textbf{MAY:} (>)
	\item \textbf{MEN:} (<)
	\item \textbf{MAYI:} ([>][=])
	\item \textbf{MENI:} ([<][=])
	\item \textbf{IGUAL:} ([=][=])
	\item \textbf{DIST:} ([!][=])
	\item \textbf{ASIG:} (=)
	\item \textbf{PAP:} ("`("')
	\item \textbf{PCIERRE:} ("`)"')
	\item \textbf{SEP:} ["` "',\textbackslash t,\textbackslash n,\textbackslash r,\textbackslash b]
		
\end{itemize}
%---------------------------------------------------------------------
\section{Dise�o}
%---------------------------------------------------------------------
\label{cap2:sec:disenyo}
El aut�mata que reconocer�a el lenguaje es el siguiente:


\end{document}
