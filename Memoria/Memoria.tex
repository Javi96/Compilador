\documentclass[11pt,oneside,onecolumn,openany]{book}

\usepackage{lmodern}
\usepackage[T1]{fontenc}
\usepackage[spanish,activeacute]{babel}
\usepackage{mathtools}

\title{Pr�ctica PL}
\author{Elena Kaloyanova Popova y �lvaro Borja Velasco Garc�a}
\date{2018}

\begin{document}
% cuerpo del documento

\maketitle
\tableofcontents % crea el �ndice

%---------------------------------------------------------------------
%                   Cap�tulo 1 - Introducci�n
%---------------------------------------------------------------------
\chapter{Introducci�n}

Esta pr�ctica consistir� en el desarrollo de un procesador de lenguajes sobre el siguiente lenguaje:

%---------------------------------------------------------------------
%                   Cap�tulo 2 - Analizador l�xico
%---------------------------------------------------------------------
\chapter{Fase 1: Analizador l�xico}

%---------------------------------------------------------------------
\section{Clases L�xicas}
%---------------------------------------------------------------------
\label{cap2:sec:clases_lexicas}
Enumeraci�n de las clases l�xicas del lenguaje. Para cada clase debe incluirse adem�s una descripci�n informal, en lenguaje natural.

%---------------------------------------------------------------------
\section{Especificaci�n Formal}
%---------------------------------------------------------------------
\label{cap2:sec:especificacion_formal}
Especificaci�n formal del l�xico del lenguaje mediante definiciones regulares.

%---------------------------------------------------------------------
\section{Dise�o}
%---------------------------------------------------------------------
\label{cap2:sec:disenyo}
Dise�o de un analizador l�xico para el lenguaje mediante un diagrama de transiciones.

\end{document}
